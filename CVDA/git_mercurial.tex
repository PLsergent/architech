\section{Git vs Mercurial}
\subsection{Format du dépôt}
\begin{frame}{Format du dépôt}
  \begin{columns}[T]

    \begin{column}{.5\textwidth}
    \textsc{Mercurial}

    \vspace{\baselineskip}

    A tout misé sur les logs en append-only, optimiser la recherche sur le
    disque de nos machines
    \end{column}

    \begin{column}{.5\textwidth}
    \textsc{Git}

    \vspace{\baselineskip}

    Stocke chaque commit/fichier dans un simple dépôt de ‘hash’ de documents,
    chaque commit finira dans ce dépôt comme une entité séparée
    \end{column}

  \end{columns}
\end{frame}

\subsection{Réécrire l'historique}
\begin{frame}{Réécrire l'historique}
  \begin{columns}[T]

    \begin{column}{.5\textwidth}
     \textsc{Mercurial}

      \begin{itemize}
        \item{Édition difficile des commits passés 
        \item{Mercurial Queus permet d’empiler des pré-commits de sorte à
              pouvoir les réorganiser jusqu’à votre commit final}
        \item{“histedit” → append-only → génère un fichier de sauvegarde externe
              problèmes: comment analyser les modifications de cette
              sauvegarde? Pendant combien de temps peut-on sauvegarder les
              sauvegardes}
      \end{itemize}
    \end{column}

    \begin{column}{.5\textwidth}
      \textsc{Git}

      \begin{itemize}
        \item{les logs de commit Git peuvent devenir des récits soigneusement 
              élaborés }
              Il est facile de « remonter le temps » pour éditer les commits
              précédents si nécessaire 
        \item{Vous arriverez au moment où vous avez besoin de faire un
              changement dans votre historique de commits; vous n’aurez
              vraiment qu’une seule chose à savoir: « interactive rebase »
              (git rebase -i) → permet de modifier l’histoire de Git comme on
              veut}
        \item{Git ne détruit pas les objets ayant une référence, pour arrêter
              le « garbage collector » de Git et retrouver une sauvegarde →
              appliquer un tag pour référence future → ce sont des commits sans
              branches}
      \end{itemize}
    \end{column}

  \end{columns}
\end{frame}

\subsection{Les branches}
\begin{frame}{Les branches}
  \begin{columns}[T]

    \begin{column}{.5\textwidth}
      \textsc{Git}

      \begin{itemize}
        \item{Namespace du serveur indique clairement qui est qui}
      \end{itemize}
    \end{column}

    \begin{column}{.5\textwidth}
      \textsc{Mercurial}

      \begin{itemize}
        \item{Extension Bookmark → clone direct des branches Git, au début on
              pouvait pas faire le push des bookmarks sur le serveur}
        \item{Problème: les bookmarks partagent le même namespace}
      \end{itemize}
    \end{column}

  \end{columns}
\end{frame}

\subsection{Staging (zone de transit)}
\begin{frame}{Staging (zone de transit)}
  \begin{columns}[T]

    \begin{column}{.5\textwidth}
      \textsc{Git}
      \begin{itemize}
        \item{Tout ce qu’on ajoute à un commit passe par cette zone de transit:
              commande “git add” → appelée aussi index}
        \item{On ajoute des modifications, pas aux fichiers eux-mêmes,
              mais au reflog}
      \end{itemize}
    \end{column}

    \begin{column}{.5\textwidth}
      \textsc{Mercurial}
      \begin{itemize}
        \item{Extension Record}
        \item{Doit copier les modifications ‘dans un emplacement temporaire →
              mettre à jour les fichiers de stockage, commit → enfin annuler
              les modifications}
        \item{Erreur = on recommence tout}
      \end{itemize}
    \end{column}

  \end{columns}
\end{frame}
